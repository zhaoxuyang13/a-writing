\documentclass[11pt]{article}
\usepackage{latexsym}
\usepackage{amsmath,amssymb,amsthm}
\usepackage{epsfig}
\usepackage[right=0.8in, top=1in, bottom=1.2in, left=0.8in]{geometry}
\usepackage{setspace}
\spacing{1.06}

\newcommand{\handout}[5]{
  \noindent
  \begin{center}
  \framebox{
    \vbox{\vspace{0.25cm}
      \hbox to 5.78in { {GE6001:\hspace{0.12cm}Scientific Writing, Norms and Ethics} \hfill #2 }
      \vspace{0.48cm}
      \hbox to 5.78in { {\Large \hfill #5  \hfill} }
      \vspace{0.42cm}
      \hbox to 5.78in { {#3 \hfill #4} }\vspace{0.25cm}
    }
  }
  \end{center}
  \vspace*{4mm}
}
\newcommand{\lecture}[4]{\handout{#1}{#2}{#3}{#4}{Notes #1}}

\newtheorem{theorem}{Theorem}
\newtheorem{corollary}[theorem]{Corollary}
\newtheorem{lemma}[theorem]{Lemma}
\newtheorem{observation}[theorem]{Observation}
\newtheorem{example}[theorem]{Example}
\newtheorem{definition}[theorem]{Definition}
\newtheorem{claim}[theorem]{Claim}
\newtheorem{fact}[theorem]{Fact}
\newtheorem{assumption}[theorem]{Assumption}
\newcommand{\E}{\textbf{E}}
\newcommand{\var}{\text{var}}
\def\eps{\ensuremath\epsilon}
\begin{document}

\lecture{1 -- Meta Learning}{May 14, 2021}{Authors:\hspace{0.08cm}\emph{Xuyang Zhao, Yuyang Huang}}


\section{Introduction}

Artificial intelligence is the simulation of human intelligence processes by machines, and especially computer systems. Through techniques such as machine learning, deep learning, neural networks,etc, computer is now able to do speech/facial recognition, natural language processing and many other tasks with ease, even outperforming mankind most of the times. 
%#
While AIs are capable of learning specific problems fast under the guidance of computer scientists and programmers, one interesting question arises, can it learn the learning process itself ?
%# 
This is what we called meta-learning. 
meta-learning is a subfield of machine learning where automatic learning algorithms are applied to metadata about machine learning process. Meta-learning tries to imitate the human learning process, that one can use previous learning experience to guide the upcoming learning event, in hope of learning more quickly or more concisely. 


With meta-learning, our AI agents should be able to learning and adapting quickly from only a few provided examples. This ability to learning fast is non-trivial but promising. Current machine learning process, especially deep learning process generally need tremendous data to reach good performance. 
But meta-learning allows the AI to learn the more general knowledge among previous tasks, so that accelerate later learning with prior knowledge, rather than learning from scratch.

\section{Problem description}

%# 定义该问题,包含各种概念。
\subsection{Machine learning}

\subsection{Few-shot learning}


\section{Algorithms}
In recent years, methods of meta learning have emerged one after another. Today we mainly introduce one of the methods, which is to learn a good initialization for the parameters, and then use a small amount of updates to train new tasks on the basis of this initialization.
According to the above introduction, we can realize that: From a macro 
perspective, meta learning uses tasks as "samples" for learning! So in 
general, we will divide the data into Meta-train and Meta-test, where 
Meta-train contains data from multiple tasks, and can be divided into D-train and D-test, which are used for training and testing, respectively.


\subsection{MAML}
Because current machine learning methods all perform gradient updates, and the focus of MAML is on gradient updates, it can also be regarded as a gradient-based meta learning method.

The core idea of MAML is actually very simple: in each iteration step, there will be an initial parameter [formula], which is used to update the gradient of K tasks using D-train and get the corresponding new parameters [formula] of different tasks, and then Use D-test on K tasks to update the global initial parameters [formula]


\begin{algorithm}
  \caption{Model-Agnostic Meta-Learning}
  \label{MAML}
  \begin{algorithmic}[1]
    \REQUIRE $p(\mathcal{T})$: distribution over tasks
    \REQUIRE $\alpha, \beta$: step size hyperparameters
    \STATE randomly initialize $\theta$
    \WHILE {not done}
    \STATE Sample batch of tasks $\mathcal{T}-i \sim p(\mathcal{T})$
    \FORALL {$\mathcal{T}-i$}
    \STATE Evaluate $\nabla-\theta \mathcal{L}-{\mathcal{T}-i} (f-\theta)$ with respect to $K$ examples
    \STATE Compute adapted parameters with gradient descent: $\theta'-i = \theta - \alpha\nabla-\theta \mathcal{L}-{\mathcal{T}-i} (f-\theta)$
    \ENDFOR
    \STATE Update $\theta \leftarrow \theta - \beta\nabla-\theta \Sigma-{\mathcal{T}-i \sim p(\mathcal{T})}\mathcal{L}-{\mathcal{T}-i} (f-{\theta'-i})$
    \ENDWHILE
  \end{algorithmic}
\end{algorithm}
\section{Key Properties}
%# 换个标题?
maybe some proof and lemma for maml, important properties.

\section{Case Study}

In this section, we will discuss how MAML algorithm will be applied to different domains. We use supervised learning and reinforcement learning as examples. These two domains differs in the form of how loss is computed and data generation methods, but basic adaptation mechanism is the same. As a model-agnostic algorithm for meta-learning, MAML offers a general approach that can also be applied to other machine learning scenarios that use gradient descent.

\subsection{MAML for Supervised Learning}

In supervised learning regime, few-shot learning is a well-studied topic,where the goal is to learn from a limited number of cases.Few-shot learning is particularly suitable for meta-learning, in that learning speed is an important meta metric of meta-learning. More specifically, cases like function regression, whose goal is to predict the outputs of a continuous-valued function with only a few points from that function, is a typical few-shot learning problem. Similarly, the cold start of user  recommendation systems is also a few-shot learning scenario, due to the limited data from specific users. Other cases including few-shot image classification also exists in practical use.

To apply MAML to supervised learning, we need to first determine the corresponding loss functions, and then replace line-5 of Algorithm-1 with case-specific model evaluation process. For the training-set and testing-set, we can just randomly sample input/output data-points for each supervised case. We use loss functions with training-set to train temporal models, and apply loss functions to testing-set to update our meta-model.

\textbf{Loss Function for Supervised Classification}
The common loss functions used for supervised classification are mean-squared error(MSE) and cross-entropy, other loss functions might also be used while they are not necessarily relevant with MAML.We use entropy loss for example below:
\begin{equation}
    \mathcal{L}_{\tau_i}(f_\phi) = \sum_{x^{(j)},y^{(j)}\sim\tau_i} y^{(j)}log{f_\phi}(x^{(j)}) + (1-y^{(j)})\log(1-f_\phi(x^{(j)}))
    \tag{3}
\end{equation}

\textbf{Loss Function for Supervised Regression}
Likewise, for few-shot regression problems, we use mean-squared error, the loss takes the form as below:
\begin{equation}
    \mathcal{L}_{\tau_i}(f_\phi) = \sum_{x^{(j)},y^{(j)}\sim\tau_i} \parallel f_\phi(x^{(j)} - y^{j})\parallel_2^2 
    \tag{4}
\end{equation}

\begin{algorithm}
  \caption{MAML for Few-Shot Supervised Learning}
  \label{MAML-supervised}
  \begin{algorithmic}[1]
    \REQUIRE $p(\mathcal{T})$: distribution over tasks
    \REQUIRE $\alpha, \beta$: step size hyperparameters
    \STATE randomly initialize $\theta$
    \WHILE {not done}
    \STATE Sample batch of tasks $\mathcal{T}_i \sim p(\mathcal{T})$
    \FORALL {$\mathcal{T}_i$}
    \STATE Sample $K$ datapoints $\mathcal{D}=\{x^{(j)}, y^{(j)}\}$ from $\mathcal{T}_i$
    \STATE Evaluate $\nabla_\theta \mathcal{L}_{\mathcal{T}_i} (f_\theta)$ using $\mathcal{D}$ and $\mathcal{L}_{\mathcal{T}_i}$ in Equation (2) or (3)
    \STATE Compute adapted parameters with gradient descent: $\theta'_i = \theta - \alpha\nabla_\theta \mathcal{L}_{\mathcal{T}_i} (f_\theta)$
    \STATE Sample datapoints $D'_i=\{x^{(j)}, y^{(j)}\}$ from $\mathcal{T}_i$ for the meta-update
    \ENDFOR
    \STATE Update $\theta \leftarrow \theta - \beta\nabla_\theta \Sigma_{\mathcal{T}_i \sim p(\mathcal{T})}\mathcal{L}_{\mathcal{T}_i} (f_{\theta'_i})$ using each $D'_i$ and $\mathcal{L}_{\mathcal{T}_i} $ in Equation 2 or 3
    \ENDWHILE
  \end{algorithmic}
\end{algorithm}


\subsection{MAML for Reinforcement learning}
In reinforcement learning, the goal of few-shot meta-learning is enabling fast reinforcement learning  on a new task using only a small amount of experience in the test setting.


\begin{algorithm}
    \caption{MAML for Few-Shot Reinforcement Learning}
    \label{MAML-reinforcement}
    \begin{algorithmic}[1]
      \REQUIRE $p(\mathcal{T})$: distribution over tasks
      \REQUIRE $\alpha, \beta$: step size hyperparameters
      \STATE randomly initialize $\theta$
      \WHILE {not done}
      \STATE Sample batch of tasks $\mathcal{T}_i \sim p(\mathcal{T})$
      \FORALL {$\mathcal{T}_i$}
      \STATE Sample $K$ trajectories $\mathcal{D}=\{(x_1,a_1,...x_H)\}$ using $f_\theta$ in $\mathcal{T}_i$
      \STATE Evaluate $\nabla_\theta \mathcal{L}_{\mathcal{T}_i} (f_\theta)$ using $\mathcal{D}$ and $\mathcal{L}_{\mathcal{T}_i}$ in Equation (4)
      \STATE Compute adapted parameters with gradient descent: $\theta'_i = \theta - \alpha\nabla_\theta \mathcal{L}_{\mathcal{T}_i} (f_\theta)$
      \STATE Sample datapoints $D'_i=\{(x_1,a_1,...x_H)\}$ using $f_{\theta'_i}$ in $\mathcal{T}_i$
      \ENDFOR
      \STATE Update $\theta \leftarrow \theta - \beta\nabla_\theta \Sigma_{\mathcal{T}_i \sim p(\mathcal{T})}\mathcal{L}_{\mathcal{T}_i} (f_{\theta'_i})$  using each $D'_i$ and $\mathcal{L}_{\mathcal{T}_i} $ in Equation 4
      \ENDWHILE
    \end{algorithmic}
  \end{algorithm}



%# need more subsections ? 

\section{Conclusion}

This paper describe a novel algorithm MAML, Model-Agnostic Meta Learning, offers a general approach to adapt meta-learning for most ML algorithms based on gradient descent. 
This methods has a lot of benefits. Firstly, it a simple and easy to understand approach to take, because it doesn't introduce any learned parameters for meta-learning. So that it can be combined with any model representation that is amenable to gradient-based training. Our case study shows that MAML can be applied to few-shot supervised learning problems like classification, regression, as well as few-shot reinforcement learn tasks. What's more, MAML is typically meta-learning the initial parameter (or a weight initialization), adaptation can be performed no matter what amount of data and what number of gradient steps. Simply applying MAML to few-shot image classification problems can already achieve state-of-the-art performance, when the original paper was published. 

\textbf{Future Work} 
Reusing knowledge from past tasks may be a crucial ingredient in making high-capacity scalable models, such as deep neural networks. 
Meta-learning is a huge field while MAML only solved the few-shot learning problems by meta-learning the initial parameters. Other metrics including robustness of ML can also be meta-learned for other cases, while other meta-learning approach like  metric-based,model-based can also be taken. With the thriving of ML academic research, learning techniques and concepts like long short-term memory, attention, neural networks, architecture search also bring opportunities to new meta-learning algorithms.
With opportunities also comes open problems. For example, diverse and multi-modal task distributions.Many big successes of meta-learning have been within narrow task families, while learning on diverse task distributions can challenge existing methods. How to learn more general knowledge just like human beings remain an interesting while challenging open problem for both computer scientists, neural scientists, philosopher and anthropologist. Let's hope meta-learning can bring mankind new understanding of themselves and their existence .



\section{Problem description}
So far, we have an algorithm $A$ which estimates in correct range of $\eps$ with probability $\ge 0.9$. Our new algorithm $A^{\ast}$ will output in range of $\eps$ with probability $1-\delta$.
Algorithm:
\begin{itemize}
\item Repeat $A$ for $m=O(log (1/\delta))$ times
\item Take median of all the $m$ answers.
\end{itemize}

To prove the correctness, we'll use Chernoff/Hoeffding bounds.

\begin{definition}
[Chernoff/Hoeffding Bound]
Let $X_{1}$, $X_{2}$, $\ldots$, $X_{m}$ be independent random variables $\in \{0,1\}$,
$\mu = E[\Sigma_{i} X_{i}], \eps \in [0,1]$.
Then $Pr[|\Sigma_{i} X_{i}-\mu| > \eps\mu] \leq 2e^{-\eps^{2}\mu/3}$
\end{definition}

Define $X_{i} = 1$ iff the $i^{th}$ answer of $A$ is correct (i.e. estimated value of $A$ lies in correct range).

\begin{claim}
$E[X_{i}] = 0.9$, and $E[\mu] = 0.9m$
\end{claim}

\begin{proof}
Since A is correct with probability 0.9, $E[X_{i}] = 0.9$. And $E[\mu] = 0.9m$ due to linearity of expectation.
\end{proof}

\begin{claim}
New algorithm $A^{\ast}$ is correct when $\Sigma_{i} X_{i} > 0.5m$
\end{claim}

\begin{proof}
Since we are considering median value to be our answer, if more than half the trials of A are correct, algorithm $A^{\ast}$ is also correct.
\end{proof}

\begin{claim}
To prove, $Pr[\Sigma_{i} X_{i} \ge 0.5m] \ge 1-\delta$ or $Pr[\Sigma_{i} X_{i} < 0.5m] < \delta$
\end{claim}

\begin{proof}
\begin{equation}
\begin{split}
Pr[\Sigma_{i} X_{i} < 0.5m] & = Pr[\Sigma_{i} X_{i} - 0.9m < -0.4m]\\
& \le Pr[|\Sigma_{i} X_{i} - \mu| > 0.4m]\\
& = Pr[|\Sigma X_{i} - \mu| > 0.4/0.9 \mu]
\end{split}
\end{equation}
Using Chernoff bound,
\begin{equation}
\begin{split}
& \leq e^{-c*0.9m}\\
& < \delta
\end{split}
\end{equation}
Above equation holds for $m = O(log(1/\delta))$
\end{proof}

\section{Distinct Elements}
Given, a stream of size $m$ containing numbers from $[n]$, we have to approximate the number of elements with non-zero frequency. To calculate the exact value the space required:

\begin{itemize}
\item $O(n)$ bits. (maintain a vector of length n).
\item $O(m \log (n))$ bits. (save m numbers, each taking $log(n)$ bits).
\end{itemize}

Since, this complexity is not feasible as $m$,$n$ can be very large, we'll look at algorithm for approximating the distinct count value.

\subsubsection{Hash Function}
\begin{itemize}
\item $h : [n] \rightarrow [0,1]$
\item $h(i)$ is uniformly distributed in $[0,1]$.
\end{itemize}

\subsection{Algorithm [Flajolet-Martin 1985]}
We maintain a variable $z$.
\begin{enumerate}
\item Initialize $z = 1$.
\item Whenever $i$ is encountered: $z = \min{(z,h(i))}$
\item When done, output $1/z -1$.
\end{enumerate}

Now, we'll prove the algorithm works in a similar fashion followed in previous lecture.
Let $d$ be number of distinct elements.

\begin{claim}
$E[z] = d+1$
\end{claim}

\begin{proof}
$z$ is the minimum of $d$ random numbers in $[0,1]$. Pick another random number $a \in [0,1]$. The probability $a<z$:
\begin{enumerate}
\item exactly z
\item probability it's smallest among $d+1$ reals : $1/(d+1)$
\end{enumerate}
Equating these two, one can prove the claim.
\end{proof}

\begin{claim}
$\text{var}[z] \leq 2/d^{2}$
\end{claim}

\begin{proof}
It can be done in a similar fashion described in previous lecture.
\end{proof}

\subsubsection{$(1+\eps)$ approximation Algorithm }
We can take $Z = (z_{1} + z_{2} + ... z_{k})/k$ for independent $z_{1}, ... z_{k}$

\subsection{Alternate Algorithm: Bottom-k}
Instead of just use the minimum value of hash function for $i$ inputs, we'll maintain the $k$ smallest hashes seen.
\begin{enumerate}
\item Initialize $(z_{1}, z_{2},...z_{k}) = 1$.
\item Keep $k$ smallest hashes seen, s.t. $z_{1}\leq z_{2}\leq...z_{k}$
\item When done, output $\hat{d} = k/z_{k}$
\end{enumerate}

\begin{claim}
The following claims are stated:
\begin{itemize}
\item $Pr[\hat{d} > (1 + \eps)d] \leq 0.05$
\item $Pr[\hat{d} < (1 - \eps)d] \leq 0.05$
\item Overall probability that $\hat{d}$ outside range is at most 0.1
\end{itemize}
\end{claim}

\begin{proof}
To compute $Pr[\hat{d} > (1+\eps)d]$:
\begin{itemize}
\item Define $X_{i} = 1$ iff $h(i) < \dfrac{k}{(1+\eps)d}$
\item Then $\hat{d} > (1+\eps)d$ iff $\Sigma_{i} X_{i} > k$
\item if $\Sigma_{i} X_{i} > k$\\
  $\iff \exists$ at least $k$ numbers for which $h(i) < \dfrac{k}{(1+\eps)d}$\\
    \begin{equation}
      \iff z_{k} < \dfrac{k}{(1+\eps)d}
      \iff \dfrac{k}{z_{k}} > (1+\eps)d
      \iff \hat{d} > (1+\eps)d
    \end{equation}
\item
  $E[X_{i}] = \dfrac{k}{(1+\eps)d}$\\
  $E[\Sigma_{i} X_{i}] = d E[X_{i}] = \dfrac{k}{1 + \eps}$\\
  $\text{var}[\Sigma_{i} X_{i}] = d \text{var}[X_{i}] \leq dE[X_{1}^{2}] \leq  \dfrac{k}{1+\eps} \leq k$\\
  (Since $X_{1} \in \{0,1\}$, $E[X_{1}^{2}] = E[X_{i}]$)
\item By Chebyshev:
    $Pr[|\Sigma X_{i} - \dfrac{k}{1+\eps}| > \sqrt{20k}] \leq 0.05 \implies Pr[\Sigma X_{i} > \dfrac{k}{1+\eps} + \sqrt{20k}] \leq 0.05 $\\
    \begin{itemize}
    \item
      (For $\eps < 1/2$ and $k=c/\eps^{2}$)\\
      $\dfrac{k}{1+\eps} + \sqrt{20k} \leq k(1-\eps+\eps^{2}) + \sqrt{20k}$ (Taylor Series Expansion)\\
      $ \leq k - k\eps/2 + 5\sqrt{c}/\eps$
      $ = k - c / 2\eps + 5\sqrt{c}/\eps$\\
      $ < k $ where $c > 100$
    \item
      Since $k > \dfrac{k}{1+\eps} + \sqrt{20k} $ in our case and $\Sigma X_{i}$ is monotonically increasing, $Pr[\Sigma X_{i} > k] \leq Pr[\Sigma X_{i} > \dfrac{k}{1+\eps} + \sqrt{20k}] \leq 0.05$

    \end{itemize}
\end{itemize}
\end{proof}

\subsection{Hash functions in stream}
The hash function we used has two practical issues: (1) the return value should be a real number. (2) how do we store it?

Discretization can solve the first issue. Instead of all the real numbers in $[0, 1]$, we use hash function with range $\{0, \frac{1}{M}, \frac{2}{M}, \frac{3}{M}, \ldots, 1\}$. For large $M \gg n^{3}$, the probability that $d \le n$ random numbers collide is at most $\frac{1}{n}$.

For the second issue, we use pairwise independent function instead of independent function.

\begin{definition}
$h: [n] \rightarrow \{1, 2, \ldots M\}$ is pairwise independent if for all $i \ne j$ and $a, b \in [M]$, $\text{Pr}[h(i)=a \land h(j)=b]=\frac{1}{M^2}$
\end{definition}

It works because in previous calculation, we only care about pairs. We defined $X_i=1$ iff $h(i)$ is small than a threshold, then we computed $\text{var}[\Sigma X_i] = E[(\Sigma X_i)^2] - E[(\Sigma X_i)^2] = E[X_1X_1 + X_1X_2 + \ldots]- E[(\Sigma X_i)^2]$. Notice that $E[X_iX_j]$ is the same for fully random $h$ and pairwise independent $h$.

\begin{example}
[Construct a pairwise independent hash]
Assume $M$ is a prime number (if not, we can always pick a larger $M$ that is a prime number). We pick $p, q \in \{0, 1, 2, \ldots M-1 \}$ and the hash function $h(i) = pi+q \mod M$. In this construction we only need $O(\log M) = O(\log n)$ space (to store $p, q, M$).
\end{example}

\begin{proof}
$h(i)=a, h(j)=b$ is equivalent to $pi+q \equiv a, pj+q \equiv b$. So $p(i-j) \equiv a-b$ and $p \equiv (a-b)(i-j)^{-1}, q \equiv a - pi$. Since $M$ is a prime number, the unique inverse implies that there is only one pair $(p, q)$ satisfies it. And the probability that pair is chosen is exactly $\frac{1}{M^2}$.
\end{proof}

\section{Impossibility Results}

We have used both approximation and randomization to solve the distinct counting problem with space much less than $\min{(m, n)}$. Now we are wondering: can we omit either approximation or randomization to achieve the same space efficiency? The answer is no.

\subsection{Deterministic Exact Won't Work}

First, we will show that there is no deterministic (no randomization) and exact (no approximation) way to solve it.

Suppose there do exists a deterministic and exact algorithm $A$ and an estimator function $R$ that use space $s \ll n, m$. That is, for a given integer stream, we first run the algorithm $A$ on the stream. As the stream goes $A$ will return middle memory steps, and we obtain the final memory state $\sigma$ after the stream ends. Then we apply $R$ on $\sigma$ to obtain our estimator $\hat{d}$. Since both $A$ and $R$ are deterministic and exact, $\hat{d}$ must equals to the distinct count for the stream.

We now build a binary representation $x$ of the stream with the following rules: (1) $x \in \{0, 1\}^{n}$, (2) $i$ in stream iff $x_i = 1$. For example, if 1, 3, 5, 6, 7 are in the stream and 2, 4 are not, $x$ will start with 1, 0, 1, 0, 1, 1, 1. Notice that each stream has a corresponding representation and streams containing different numbers have different representations.

\begin{claim}
We can recover the $x$ of the stream given the memory state $\sigma$
\end{claim}

\begin{proof}
Denote $d=R(\sigma)$ be the original estimator. Now we treat $\sigma$ as a middle snapshot of the memory and add integer $i$ as the next element of the stream. Now $A$ will return another memory state $\sigma'$, and $d'=R(\sigma)'$ will be our new estimator. If $d'=d$, $i$ must have appeared in the stream before since $A$ and $R$ are deterministic and exact. Similarly, if $d'>d$, $i$ must have not appeared in the stream before. Using this method with $i=1, 2, 3\ldots$ and we can recover the $x$.
\end{proof}

Since we can recover $x$ from $\sigma$, we can treat $\sigma$ as an encoding of a string $x$ of length $n$. But $\sigma$ has only $s \ll n$ bits! Furthermore, we can treat $A$, the function that produces $\sigma$, as a function with domain $\{0, 1\}^{n}$ and $\{0, 1\}^{s}$. We can see that $A$ must be injective because if $A(x)=A(x')=\sigma$, the recoverability implies $x=x'$.

Hence $s \ge n$. Which implies that there is no deterministic and exact algorithm $A$ and an estimator function $R$ that use space $s \ll n, m$.

\subsection{Deterministic Approx. Won't Either}

We can use the similar strategy to prove that deterministic approx. won't work. We pick $T \subset \{0, 1\}^{n}$ that satisfies the following conditions: (1) for all distinct $x, y \in T$, the number of digits $i$ that $y_i=1$ and $x_i=0$ should $\ge \frac{n}{6}$. (2) $|T| \ge 2^{\Omega(n)}$. Now we use algorithm $A$ to encode an input $x$ into $\sigma=A(x)$ and our estimator would be $\hat{d}=R(\sigma)$.

Now we want to recover $x$ based on $\sigma$, as what we have done in the last section. For a given $\sigma$ and any $y \in T$, we append $y$ to the stream and apply $A$ on it, and $A$ will return a memory state $\sigma'$. Using $\sigma'$ we have new estimator $\hat{d'}=R(\sigma')$.

\begin{claim}
If $\hat{d'} > 1.01 \hat{d}$, then $x \ne y$, else $x=y$.
\end{claim}

\begin{proof}
The idea is that when $x=y$, $\hat{d}$ would be really close to $\hat{d'}$ (up to $(1+\epsilon)^{2}$ because both of them are $\epsilon$-approximated) and when $x \ne y$, the construction of $T$ guarantee that $\hat{d} \ge \hat{d} + \frac{n}{6}$. So we can pick an $\epsilon$ that works for our claim.
\end{proof}

We can use this method to check every element $y \in T$ to see if $y=x$, and eventually we can recover $x$ from it. Similar to last section, we can show that $A$ is an injective function and it implies that $2^{s} \ge |T|$ or $s = \Omega(n)$.

\section{Concluding Remarks}

\begin{itemize}

\item We can use median trick and Chernoff bound to improve the probability of an existing algorithm.

\item For distinct elements problem, we can also store the hashes $h(i)$ approximately. One example is to store the number of leading zeros, and it only cost $O(\log \log n)$ bits per hash value, and that is the idea behind another algorithm called HyperLogLog.

\item For the impossibility results, we can also prove that randomized exact algorithm won't work.

\end{itemize}


\end{document}
